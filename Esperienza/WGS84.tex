\documentclass[11pt, a4paper]{article}
\usepackage[italian]{babel}
\usepackage[utf8]{inputenc}
\usepackage[T1]{fontenc}
\usepackage{geometry}
\usepackage{amsmath}
\usepackage{amssymb}
\usepackage{array}
\usepackage{booktabs}
\usepackage{siunitx}
\usepackage{graphicx}
\usepackage{parskip}
\usepackage{hyperref}
\hypersetup{
    colorlinks=true,
    linkcolor=blue,
    urlcolor=blue,
}

\geometry{margin=2.5cm}
\title{\textbf{WGS-84: World Geodetic System 1984}}
\author{}
\date{}

\begin{document}

\maketitle

\section*{Introduzione}
WGS-84 è il \textbf{sistema geodetico di riferimento mondiale} adottato dal Dipartimento della Difesa degli Stati Uniti nel 1984 e tuttora usato dal GPS e dalla maggior parte delle carte nautiche e aree globali.

\section{Cos'è un sistema geodetico?}
Un sistema geodetico è un insieme di:
\begin{itemize}
    \item un \textbf{ellissoide di riferimento} (forma e dimensioni della Terra),
    \item un \textbf{datum} (posizione dell'ellissoide rispetto al centro di massa della Terra),
    \item un sistema di coordinate (latitudine, longitudine, altezza),
    \item un modello di campo gravitazionale associato.
\end{itemize}

\section{Parametri principali di WGS-84}
\begin{table}[h!]
    \centering
    \begin{tabular}{lcc}
        \toprule
        \textbf{Grandezza} & \textbf{Simbolo} & \textbf{Valore} \\
        \midrule
        Semiasse maggiore & $a$ & \SI{6 378 137,0}{\meter} \\
        Semiasse minore & $b$ & \SI{6 356 752,314245}{\meter} \\
        Schiacciamento & $f = \dfrac{a - b}{a}$ & $1 / 298,257\,223\,563$ \\
        Costante geocentrica gravitazionale & $GM$ & \SI{3,986004418e14}{\meter\cubed\per\second\squared} \\
        Velocità angolare terrestre & $\omega$ & \SI{7,292115e-5}{\radian\per\second} \\
        \bottomrule
    \end{tabular}
\end{table}

\section{Cosa fornisce WGS-84?}
\begin{itemize}
    \item \textbf{Coordinate GPS}: latitudine e longitudine espressi proprio sul suo ellissoide.
    \item \textbf{Altezza ellisoidica} ($h$): distanza lungo la normale all'ellissoide, non l'altezza sul livello del mare (quella si chiama \emph{altezza ortometrica}).
    \item \textbf{Modello di gravità normale}: la formula di Somigliana (quella che si usa per $g$ teorico) è parte dello standard WGS-84.
    \item \textbf{Frame di riferimento terrestre}: il centro dell'ellissoide coincide con il centro di massa della Terra ($\pm 2$ cm).
\end{itemize}

\section{Perché è importante nel tuo esperimento?}
\begin{itemize}
    \item Il GPS del telefono restituisce \textbf{latitudine $\varphi$} e \textbf{altezza ellisoidica $h$} proprio in WGS-84.
    \item La formula che usi per calcolare $g$ teorico è la \textbf{``gravity formula 1984''} del WGS-84: senza quei parametri il confronto con la misura non sarebbe coerente.
\end{itemize}

\section{WGS-84 vs altri sistemi}
\begin{table}[h!]
    \centering
    \begin{tabular}{lcc}
        \toprule
        \textbf{Sistema} & \textbf{Uso tipico} & \textbf{Differenza rispetto a WGS-84} \\
        \midrule
        ETRS89 & Europa & $\sim 0,5$--1 m (fissa all'Europa, diverge 2,5 cm/anno) \\
        ED50 & Carte nautiche europee storiche & decine di metri \\
        ITRFxx & Ricerca geodetica & centimetri, aggiornato ogni anno \\
        \bottomrule
    \end{tabular}
\end{table}

\noindent Per un esperimento di fisica liceale queste differenze sono trascurabili; usare WGS-84 è sufficiente e coerente con i dati GPS del telefono.

\end{document}